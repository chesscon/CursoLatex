\documentclass{article}

\usepackage[utf8]{inputenc}
\usepackage[spanish, mexico]{babel}

\title{Crédito}

\begin{document}

\maketitle

\section{Etimología}
La palabra \textbf{crédito} proviene del latín \emph{credititus} (sustantivación del verbo \emph{credere}: creer), que significa ``cosa confiada''. Así "crédito" en su origen significa entre otras cosas, confiar o tener confianza. Se considerará crédito, el derecho que tiene una persona acreedora a recibir de otra deudora una cantidad en numerario para otros. En general es el cambio de una riqueza presente por una futura, basado en la confianza y solvencia que se concede al deudor. El crédito, según algunos economistas, es una especie de cambio que actúa en el tiempo en vez de actuar en el espacio. Puede ser definido como \textit{``el cambio de una riqueza presente por una riqueza futura''}. Así, si un molinero vende 100 sacos de trigo a un panadero, a 90 días plazo, significa que confía en que llegada la fecha de dicho plazo le será cancelada la deuda. En este caso se dice que la deuda ha sido ``a crédito, a plazo''. En la vida económica y financiera, se entiende por crédito, por consiguiente, la confianza que se tiene en la capacidad de cumplir, en la posibilidad, voluntad y solvencia de un individuo, por lo que se refiere al cumplimiento de una obligación contraída.

\section{Crédito revolvente}

Los clientes de tarjetas de crédito pueden tener diferentes formas para pagar el uso de su línea de crédito. Por lo general será en cuotas o en modalidad \emph{revolving}. Los clientes que tienen modalidad \emph{revolving} pueden realizar un pago menor al total facturado en el período (llamado Pago Mínimo). El saldo (la diferencia entre lo facturado y lo pagado), genera una nueva deuda (\emph{revolving}) a la que se le aplica la tasa de interés vigente para el período y se adiciona al saldo de deuda de esta modalidad, correspondientes a los períodos anteriores si existieren. Esta deuda puede ser pagada (amortizada) por el cliente de manera diferida en el tiempo.

\section{Tipos de créditos}

\begin{itemize}
\item \textbf{Crédito tradicional}: Préstamo que contempla un pie y un número de cuotas a convenir. Habitualmente estas cuotas incluyen seguros ante cualquier siniestro involuntario.
\item \textbf{Crédito al consumo}: Préstamo a corto o mediano plazo (1 a 4 años) que sirve para adquirir bienes o cubrir pago de servicios.
\item \textbf{Crédito comercial}: Préstamo que se realiza a empresas de indistinto tamaño para la adquisición de bienes, pago de servicios de la empresa o para refinanciar deudas con otras instituciones y proveedores de corto plazo.
\item \textbf{Crédito hipotecario}: Dinero que entrega el banco o financiera para adquirir una propiedad ya construida, un terreno, la construcción de viviendas, oficinas y otros bienes raíces, con la garantía de la hipoteca sobre el bien adquirido o construido; normalmente es pactado para ser pagado en el mediano o largo plazo (8 a 40 años, aunque lo habitual son 20 años).
\item \textbf{Crédito consolidado}: Es un préstamo que reúne todos los otros préstamos que un prestatario tiene en curso, en un único y nuevo crédito. Habitualmente estos préstamos consolidados permiten a quienes los suscriben pagar una cuota periódica inferior a la suma de las cuotas de los préstamos separados, si bien en contraprestación suele prolongarse el plazo del crédito y/o el tipo de interés a aplicar.
\item \textbf{Crédito personal}: Dinero que entrega el banco o financiera a un individuo , persona física, y no a personas jurídica,para adquirir un bien mueble (entiéndase así por bienes que no sean propiedades/viviendas), el cual puede ser pagado en el mediano o corto plazo (1 a 6 años).
\item \textbf{Crédito prendario}: Dinero que le entrega el banco o entidad financiera a una persona física, y no a personas jurídicas para efectuar la compra de un bien mueble, generalmente el elemento debe de ser aprobado por el banco o entidad financiera, y puesto que este bien mueble a comprar quedara con una prenda, hasta una vez saldada la deuda con la entidad financiera o Bancaria.
\item \textbf{Crédito rápido}: Es un tipo de préstamo que suelen comercializar entidades financieras de capital privado, de baja cuantía y cierta flexibilidad en los plazos de amortización, convirtiéndose en productos atractivos sobre todo en casos de necesidades urgentes de liquidez.
\item \textbf{Mini Crédito}: Préstamo de baja cuantía (hasta 600 euros) a devolver en no más de 30 días que conceden las entidades de crédito. Se caracterizan por su solicitud ágil, su aprobación o denegación rápidas y por ser bastante más caros que los préstamos bancarios. 

\end{itemize}

\noindent \textsc{Fuente:} \texttt{http://es.wikipedia.org/wiki/Credito}

\end{document}