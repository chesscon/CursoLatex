\documentclass[letterpaper,12pt]{book}

\usepackage[utf8]{inputenc} % Soporte para acentos
\usepackage[T1]{fontenc}    
\usepackage[spanish,es-tabla, mexico]{babel} % Español
\usepackage{amsmath}		% Soporte de símbolos adicionales (matemáticas)
\usepackage{amssymb}		
\usepackage{amsfonts}
\usepackage{latexsym}

% Información para el título
\title{Expresiones Matemáticas 2}
\author{J. Luis Torres}

% Indicamos una separación entre los párrafos
\parskip=3mm

% Eliminamos la sangría de los párrafos
\parindent=0mm

%\pagestyle{myheadings}
%\markright{\LaTeX \hfill Expresiones matemáticas 2\;\;}

\begin{document}

\maketitle
\frontmatter
\tableofcontents

\chapter{Prólogo}

En esta sección se debe colocar la información del prólogo.

El comando \verb@\frontmatter@ le indica a \LaTeX{} que estas secciones deben numerarse de manera distinta,
generalmente con números romanos. Esto se puede utilizar para colocar la numeración al prólogo, 
tabla de contenidos, índice de figuras, índice de tablas, etcétera.

El comando \verb@\mainmatter@ que insertamos más abajo indica que se debe iniciar una 
numeración con arábigos. Después de este comando se coloca el contenido del documento, es decir, capítulos.

Al final del libro se puede utilizar el comando \verb@\appendix@ para colocar anexos, solamente se incluye antes del primer anexo. 

También se puede utilizar \verb@\pagenumbering{estilo}@ para indicar el tipo de numeración a usar. Las opciones
de estilo son: 

\begin{itemize}
	\item \textbf{arabic}: números arábigos. 
	\item \textbf{roman}: números romanos con letras minúsculas. 
	\item \textbf{Roman}: números romanos con letras mayúsculas.
	\item \textbf{alph}: letras minúsculas. 
	\item \textbf{Alph}: letras mayúsculas.
\end{itemize}

Otro comando que podemos usar es \verb@\setcounter{page}{número}@ para indicar dónde se debe iniciar la numeración
en un capítulo. Este comando se coloca inmediatamente después de \verb@\chapter{}@.

\mainmatter

\chapter{Elementos Espaciales}

\section{Arreglos y Matrices}

Los arreglos se pueden introducir con el entorno \textbf{array}, siguiendo estos pasos\footnote{\textbf{Es importante considerar que este entorno se debe usar en modo matemático.}}

\begin{itemize}
	\item Indicamos el inicio del entorno: \verb@\begin{array}{alineación}@
	\item Podemos indicar la alineación de las columnas en el indicador de inicio del entorno,
		colocando las letras \textbf{l} para izquierda, \textbf{c} para centrado y \textbf{r} para derecha.\\
		Por ejemplo: \verb@\begin{array}{lccr}@ indica un arreglo de cuatro columnas con la primera alineada
		a la izquierda, la segunda y tercera centradas, y la cuarta alineada a la derecha.
	\item Colocamos la información de las columnas, separando el contenido de éstas con el símbolo \textbf{\&}.
		También debemos indicar el fin de cada renglón con un salto de linea \verb@\\@.
	\item Colocamos el finalizador del entorno: \verb@\end{array}@.
\end{itemize}

Veamos algunos ejemplos:

\[
\begin{array}{lcc}
	\mbox{Ecuación 1} & x^2 + 2x + 3 \\
	\mbox{Ecuación 2} & sen(x^3) + cos(x^2) \\
	\mbox{Ecuación 3} & x^3 - x^2
\end{array}
\]

\emph{Nota:} en el modo matemático se pueden introducir acentos usando el comando \verb@\acute{}@, pero si una
expresión se introduce por medio de \verb@\mbox{}@ los acentos pueden introducirse normalmente.

\[
\begin{array}{ccccc}
	\alpha & 5 & 2.35 & sen(x) & \beta \\
	2.3 & 0 & 4.5 & & \pi \\
	3 & 7 & 6 & 8 & \sqrt[5]{2 \phi^2 + \beta^3} \\
	\displaystyle\sum_{0}^{n} x^n & \infty & 6.4 & -9 & 2 \pi \\
	-9.5 & 2.3 & 17.8 & 7676476 & \frac{75 x}{x-1}
\end{array}
\]

En este tipo de expresiones podemos colocar los delimitadores que se han visto previamente, por ejemplo:

\[
Arreglo_1 = \left\{\begin{array}{ccccc}
	\alpha & 5 & 2.35 & sen(x) & \beta \\
	2.3 & 0 & 4.5 & & \pi \\
	3 & 7 & 6 & 8 & \sqrt[5]{2 \phi^2 + \beta^3} \\
	\displaystyle\sum_{0}^{n} x^n & \infty & 6.4 & -9 & 2 \pi \\
	-9.5 & 2.3 & 17.8 & 7676476 & \frac{75 x}{x-1}
\end{array}
\right\}
\]

\[
Arreglo_2 = \left[ \begin{array}{ccccc}
	\alpha & 5 & 2.35 & sen(x) & \beta \\
	2.3 & 0 & 4.5 & & \pi \\
	3 & 7 & 6 & 8 & \sqrt[5]{2 \phi^2 + \beta^3} \\
	\displaystyle\sum_{0}^{n} x^n & \infty & 6.4 & -9 & 2 \pi \\
	-9.5 & 2.3 & 17.8 & 7676476 & \frac{75 x}{x-1}
\end{array}
\right]
\]

Podemos hacer uso de los delimitadores \verb@\left(@ y \verb@\right)@ para introducir matrices, haciendo uso
de arreglos. 

Veamos algunos ejemplos:

\[
A = \left( 
\begin{array}{ccccc}
	\alpha & 5 & 2.35 & sen(x) & \beta \\
	2.3 & 0 & 4.5 & & \pi \\
	3 & 7 & 6 & 8 & \sqrt[5]{2 \phi^2 + \beta^3} \\
	\displaystyle\sum_{0}^{n} x^n & \infty & 6.4 & -9 & 2 \pi \\
	-9.5 & 2.3 & 17.8 & 7676476 & \frac{75 x}{x-1}
\end{array}
\right)
\]

\[
B = \left( 
\begin{array}{lcc}
	a & a+b & b-a \\
	b & b & k-a-b \\
	\vdots & \vdots & \vdots \\
	c/a & \sqrt{b} + d & 2d
\end{array}
\right)
\]

\[
C = \left( 
\begin{array}{lccc}
	a & a+b & \hdots & b-a \\
	b & b & \hdots & k-a-b \\
	\vdots & \vdots & \hdots & \vdots \\
	c/a & \sqrt{b} + d & \hdots & 2d
\end{array}
\right)
\]

\[
D = \left( 
\begin{array}{lccc}
	a & a+b & \cdots & b-a \\
	b & b & \cdots & k-a-b \\
	\vdots & \vdots & \ddots & \vdots \\
	c/a & \sqrt{b} + d & \hdots & 2d
\end{array}
\right)
\]

También podemos hacer uso de arreglos para incluir, por ejemplo, funciones definidas por partes:

\[
f(x)= \left\{ 
\begin{array}{lcl}
	x^2 + 2 & \mbox{ si } & x<0 \\
	& & \\
	x-1 & \mbox{ si } & 0 < x < 2 \pi \\
	& & \\
	2 x - x^2 & \mbox{ si } & x>2 \pi
\end{array}
\right.
\]

Los arreglos pueden contener arreglos. Por ejemplo:

\[
f(x)= \left\{ 
\begin{array}{lcl}
	x^2 + 2 & \mbox{ si } & x<0 \\
	& & \\
	(2x \, \frac{1}{(x+1)^2})\left(\begin{array}{c} \frac{x}{sin(x)} \\ 4x \end{array}\right) & \mbox{ si } & 0 < x < 2 \pi \\
	& & \\
	2 x - x^2 & \mbox{ si } & x>2 \pi
\end{array}
\right.
\]

Podemos indicar un producto de matrices colocando simplemente una a continuación de la otra. Por ejemplo:

\[
B \cdot C = \left( 
\begin{array}{lcc}
	a & a+b & b-a \\
	b & b & k-a-b \\
	\vdots & \vdots & \vdots \\
	c/a & \sqrt{b} + d & 2d
\end{array}
\right)\left( 
\begin{array}{lccc}
	a & a+b & \hdots & b-a \\
	b & b & \hdots & k-a-b \\
	\vdots & \vdots & \hdots & \vdots \\
	c/a & \sqrt{b} + d & \hdots & 2d
\end{array}
\right)
\]

El paquete \emph{amsmath} contiene varios entornos que pueden ser usados para introducir matrices. Por ejemplo:

El entorno \textbf{matrix} (arreglos):
\[
\begin{matrix}
	1 & 2 \\
	3 & 4
\end{matrix}
\]
El entorno \textbf{pmatrix} (matrices):
\[
\begin{pmatrix}
	1 & 2 \\
	3 & 4
\end{pmatrix}
\]
El entorno \textbf{bmatrix} (con corchetes):
\[
\begin{bmatrix}
	1 & 2 \\
	3 & 4
\end{bmatrix}
\]

\pagebreak

\section{Ecuaciones Multilínea}

Para introducir ecuaciones que ocupan varias lineas podemos hacer uso del entorno \verb@aligned@ o 
\verb@split@ dentro del modo matemático. Por ejemplo:

\[
\begin{aligned}
	M_{1,1} &= a_{1,1} \cdot b_{1,1} + a_{1,2} \cdot b_{2,1} + a_{1,3} \cdot b_{3,1} \\
			&= x^2 \cdot 2x + x \cdot 2x + \frac{1}{x} \cdot sin(x) \\
			&= 4 \cdot 4 + 2 \cdot 4 + \frac{1}{2} \cdot sin(2) 
\end{aligned}
\]

Este tipo de expresiones también pueden introducirse mediante el entorno \verb@equation@.

\begin{equation*}
\begin{aligned}
	M_{1,1} &= a_{1,1} \cdot b_{1,1} + a_{1,2} \cdot b_{2,1} + a_{1,3} \cdot b_{3,1} \\
			&= x^2 \cdot 2x + x \cdot 2x + \frac{1}{x} \cdot sin(x) \\
			&= 4 \cdot 4 + 2 \cdot 4 + \frac{1}{2} \cdot sin(2) 
\end{aligned}
\end{equation*}

También pueden colocarse mediante el entorno \verb@align@.

\begin{align*}
	M_{1,1} &= a_{1,1} \cdot b_{1,1} + a_{1,2} \cdot b_{2,1} + a_{1,3} \cdot b_{3,1} \\
			&= x^2 \cdot 2x + x \cdot 2x + \frac{1}{x} \cdot sin(x) \\
			&= 4 \cdot 4 + 2 \cdot 4 + \frac{1}{2} \cdot sin(2) 
\end{align*}

Nótese que también es posible hacer uso de arreglos para introducir ecuaciones de varias lineas.

\newpage

\section{Arreglos de Ecuaciones}

Los arreglos de ecuaciones se pueden introducir mediante los entornos \verb@aligned@ o \verb@align@. Por ejemplo:

\begin{equation*}
\begin{aligned}
	x^2 + 2x + 3 &= 0 \\
	3x^2 - x + 4 &= 0 \\
	-x^2 + x - 3 &= 0
\end{aligned}
\end{equation*}

\begin{align*}
	x^2 + 2x + 3 &= 0 \\
	3x^2 - x + 4 &= 0 \\
	-x^2 + x - 3 &= 0
\end{align*}

Aparentemente obtenemos el mismo resultado, pero veamos que sucede si colocamos estos entornos sin asterisco:

\begin{equation} \label{sist:eq1}
\begin{aligned}
	x^2 + 2x + 3 &= 0 \\
	3x^2 - x + 4 &= 0 \\
	-x^2 + x - 3 &= 0
\end{aligned}
\end{equation}

\begin{align}
	x^2 + 2x + 3 &= 0 \\
	3x^2 - x + 4 &= 0 \\
	-x^2 + x - 3 &= 0
\end{align}

En el entorno \verb@align@ podemos indicar que una de las ecuaciones no debe numerarse, mediante el comando \verb@\nonumber@:

\begin{align}
	x^2 + 2x + 3 &= 0 \\
	3x^2 - x + 4 &= 0 \nonumber \\
	-x^2 + x - 3 &= 0
\end{align}

También podemos insertar lineas de texto entre las ecuaciones de
nuestro arreglo:

\begin{align}
	x^2 + 2x + 3 &= 0 \\
	\intertext{la siguiente ecuación no es parte del sistema}
	3x^2 - x + 4 &= 0 \nonumber \\
	-x^2 + x - 3 &= 0
\end{align}

Otra opción para colocar ecuaciones muy grandes es hacer uso del
entorno \verb@multline@.

\begin{multline}
 \sum_{i=1}^{15} x_i = x_1 + x_2 + x_3 + x_4 + x_5 + x_6 + x_7 + 	
 	x_8 + x_9 + \\
 	 x_{10} + x_{11} + x_{12} + x_{13} + x_{14} + x_{15}
\end{multline}

\section{Numeración y Referencia de Ecuaciones}

Las ecuaciones y arreglos de ecuaciones pueden colocarse por medio de arreglos, pero existe una ventaja
al hacer uso de entornos como \verb@equation@, \verb@align@ o \verb@aligned@, éstos últimos colocan una
numeración de manera automática a las ecuaciones.

\begin{equation}
	E = m\cdot C^2
\end{equation}

Podemos hacer uso de esta numeración para referenciar las ecuaciones, agregando etiquetas a éstas mediante
el comando \verb@\label{}@.

\begin{equation} \label{eq:Einstein}
	E = m\cdot C^2
\end{equation}

Después podemos hacer referencia a una ecuación a través de su etiqueta y \LaTeX{} automáticamente insertará
el número de ecuación correspondiente\footnote{Cuando se usan este tipo de referencias el documento debe 
compilarse dos veces.}.

Si desea trabajar con una ecuación sencilla revise \eqref{eq:Einstein}.

Tarea 1: resuelva el sistema de ecuaciones de \eqref{sist:eq1}.

\newpage

\chapter{Tablas}

\section{Creación de tablas}

El entorno \verb@tabular@ nos permite crear tablas en \LaTeX{}.

La sintaxis es muy similar a los entornos \verb@array@ y 
\verb@pmatrix@, pero en este caso debemos indicar la forma
en la que se deben colocar las lineas divisorias. Por ejemplo:

\begin{tabular}{|c|c|c|} \hline
$p$ & $q$ & $p \rightarrow q$\\\hline
0 & 0 & 1 \\
0 & 1 & 1 \\
1 & 0 & 0 \\
1 & 1 & 1 \\\hline
\end{tabular}

En general, la sintaxis es la siguiente:

\begin{itemize}
	\item Indicamos el inicio del entorno: 
		\verb@\begin{tabular}{formato de columnas}@
		
		El formato de las columnas lo indicamos de la misma forma
		que en el caso de los arreglos, incluyendo además las
		lineas divisorias, con el caracter \textbf{|}.
	\item Con el comando \verb@\hline@ indicamos si se debe
		colocar una linea horizontal en la parte superior.
	\item Colocamos la información de cada renglón de la tabla,
		separando cada columna con \textit{ampersand}.
	\item Al final de cada renglón incluimos un salto de linea y
		podemos indicar si se debe colocar una linea horizontal
		antes de iniciar el siguiente renglón.
	\item Indicamos el final del entorno con \verb@\end{tabular}@.
\end{itemize}

El entorno \verb@tabular@ puede ser incluido dentro del entorno
\verb@table@ para agregar algunas características a la tabla.
Por ejemplo:

\begin{table}
\begin{tabular}{|c|c|c|} \hline
$p$ & $q$ & $p \rightarrow q$\\\hline
0 & 0 & 1 \\
0 & 1 & 1 \\
1 & 0 & 0 \\
1 & 1 & 1 \\\hline
\end{tabular}
\end{table}

Comparemos con la siguiente tabla:

\begin{table}[h!]
\centering
\begin{tabular}{|c|c|c|} \hline
$p$ & $q$ & $p \rightarrow q$\\\hline
0 & 0 & 1 \\
0 & 1 & 1 \\
1 & 0 & 0 \\
1 & 1 & 1 \\\hline
\end{tabular}
\caption{Tabla de verdad $p \rightarrow q$}
\end{table}

El comando \verb@\hline@ nos permite colocar una linea horizontal
del ancho de la tabla. También podemos hacer uso del comando
\verb@\cline{i-j}@ para colocar una linea de la columna i a la j.
Por ejemplo:

\begin{table}[h!]
\centering
\begin{tabular}{| l | c | r | }
	\hline
   1 & 2 & 3 \\
   \cline{1-2}
   4 & 5 & 6 \\
   \cline{2-2}
   7 & 8 & 9 \\
   \hline
 \end{tabular}
\end{table}

\appendix
\chapter{Algunos símbolos}

\section{Símbolos relacionales}

\begin{equation*}
\begin{aligned}
\leq & & \verb@\leq@ \\
\geq &	& \verb@\geq@ \\
\succ &	& \verb@\succ@ \\
\succeq & & \verb@\succeq@ \\
\gg	 &	& \verb@\gg@ \\
\ll & & \verb@\ll@ \\
\subset & & \verb@\subset@ \\
\subseteq & & \verb@\subseteq@ \\
\sqsubseteq & & \verb@\sqsubseteq@ \\
\supset & & \verb@\supset@ \\
\supseteq & & \verb@\supseteq@ \\
\sqsupseteq & & \verb@\sqsupseteq@ \\
\in & & \verb@\in@ \\
\ni & & \verb@\ni@ \\
\parallel & & \verb@\parallel@ \\
\approx & & \verb@\approx@
\end{aligned}
\end{equation*}

\end{document}
